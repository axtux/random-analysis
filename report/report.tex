\documentclass[10pt,a4paper]{article}
\usepackage[utf8]{inputenc}
\usepackage[french]{babel}
\usepackage[T1]{fontenc}
\usepackage{graphicx}
\usepackage{listings}

\usepackage{fancyhdr}
\usepackage{vmargin}

\setlength{\parindent}{0cm}
\setlength{\parskip}{1ex plus 0.5ex minus 0.2ex}
\newcommand{\hsp}{\hspace{20pt}}
\newcommand{\HRule}{\rule{\linewidth}{0.5mm}}

\begin{document}
	\pagestyle{fancy}
	\fancyhf{}
	\rhead{SALEMI Marco, LECOCQ Alexis}
	\lhead{Pi randomness}
	\cfoot{\thepage}
	
	\begin{titlepage}
		\begin{sffamily}
			\begin{center}
				% Upper part of the page. The '~' is needed because \\
				% only works if a paragraph has started.
				\includegraphics[scale=1.5]{images/pi.png}~\\[1.5cm]
				
				% Title
				\HRule \\[0.5cm]
				{ \huge \bfseries Pi randomness\\[0.4cm] }
				\HRule \\[1.5cm]
				
				\Large{Rapport de projet de simulation}\\[2cm]
				
				\Large{Année académique 2016-2017}\\[2cm]
				
				% Author and supervisor
				\begin{minipage}{0.4\textwidth}
					\begin{flushleft} \large
						\emph{\textbf{Auteurs :}}\\
						SALEMI Marco\\
						LECOCQ Alexis
					\end{flushleft}
				\end{minipage}
				\begin{minipage}{0.4\textwidth}
					\begin{flushright} \large
						\emph{\textbf{Directeurs :}}\\
						BUYS Alain\\
						-\\
					\end{flushright}
				\end{minipage}
				
				\vfill
				
				% Bottom of the page
				{\large \today}
				
			\end{center}
		\end{sffamily}
	\end{titlepage}
	
	\newpage
	\tableofcontents
	
	\newpage
	\section{Introduction}
	Dans le cadre du cours de simulation, nous avons été amenés à réaliser un projet afin de mettre en pratique la théorie vue au cours.
	
	Les objectifs du projet sont :
	\begin{enumerate}
		\item analyser le caractère aléatoire des décimales de pi par des tests vus au cours ;
		\item utiliser ces décimales pour construire un générateur de loi uniforme dans l'intervalle [0, 1[ ;
		\item comparer le générateur du point 2 avec celui utilisé par défaut dans Python.
	\end{enumerate}
	
	Pour ce faire, un fichier nous est fourni. Celui-ci contient les 1 000 000 premières décimales du nombre pi.
	
	Le projet doit être réalisé en python et nous avons opté pour la version 3.
	
	\newpage
	\section{Les décimales de pi}
	
	\subsection{Test de khi2}
	Le premier test consiste à étudier le nombre d'apparitions de chaque décimale. Le résultat aléatoire attendu, c'est que l'ensemble des décimales apparaissent exactement le même nombre de fois.
	
	\newpage
	\section{Conclusion}
	Nous avons bien réalisé les objectifs fixés dans l'introduction, à savoir analyser le caractère aléatoire des décimales de pi, construire un générateur uniforme et le comparer au générateur par défaut de Python.
	
	Nous avons ainsi eu l'occasion de mettre en pratique et d'approfondir les concepts vus au cours théorique notamment les test de khi2, ...
	% TODO complete tests
	
	Nous tenons à remercier le titulaire BUYS Alain pour le dévouement dont il a fait preuve cette année.
	
\end{document}